\newpage
\chapter{METODE PENELITIAN} \label{Bab III}

\section{Alur Penelitian} \label{III.Alur}
Penelitian ini menggunakan pendekatan eksperimental komparatif untuk mengevaluasi kinerja tiga generasi model pembelajaran mesin dalam klasifikasi \textit{Functional Requirements} (FR) dan \textit{Non-Functional Requirements} (NFR). Proses penelitian dirancang untuk memastikan validitas internal melalui kontrol variabel yang ketat dan validitas eksternal melalui penggunaan dataset yang representatif.

Tahapan penelitian meliputi:
\begin{enumerate}
    \item Konstruksi dataset mandiri dengan total 1.000 sampel.
    \item Pra-pemrosesan data adaptif berdasarkan arsitektur model.
    \item Rekayasa fitur: TF-IDF, \textit{trainable embedding}, dan \textit{contextual embedding}.
    \item Pelatihan tiga model (SVM, Bi-LSTM, DistilBERT) dengan strategi penanganan ketidakseimbangan kelas.
    \item Evaluasi numerik menggunakan \textit{Stratified 5-Fold Cross Validation}.
    \item Analisis kesalahan dan interpretabilitas model.
\end{enumerate}

Diagram alir metodologi ditunjukkan pada Gambar \ref{fig:3.alur}.

\begin{figure}[!ht]
    \centering
    \includegraphics[width=0.95\textwidth, height=0.85\textheight, keepaspectratio]{figure/diagram.png}
    \caption{Diagram Alir Tahapan Penelitian}
    \label{fig:3.alur}
\end{figure}

\section{Konstruksi Dataset} \label{III.Dataset}

\subsection{Sumber Data}
Dataset dikumpulkan dari berbagai sumber untuk menjamin keragaman linguistik, yang meliputi:
\begin{enumerate}
    \item Dokumen Spesifikasi Kebutuhan Perangkat Lunak (SKPL).
    \item Laporan Tugas Akhir mahasiswa bidang Informatika.
    \item Dokumen kebutuhan proyek perangkat lunak \textit{open-source} dari repositori publik.
\end{enumerate}

\subsection{Statistik Dataset}
Total dataset yang dikurasi berjumlah \textbf{1.000 kalimat}, dengan rincian distribusi kelas sebagai berikut:
\begin{enumerate}
    \item \textbf{FR}: 580 kalimat (58\%).
    \item \textbf{NFR}: 420 kalimat (42\%).
    \item \textbf{Total}: 1.000 kalimat.
    \item Rata-rata panjang kalimat: 17 token.
    \item Panjang maksimum: 55 token.
\end{enumerate}

Jumlah data ini (1.000 sampel) dianggap memadai untuk proses \textit{fine-tuning} model bahasa modern dan pelatihan model \textit{Deep Learning} sederhana tanpa risiko \textit{overfitting} yang signifikan, asalkan disertai teknik regularisasi yang tepat. Distribusi kelas mengikuti karakteristik alami dokumen kebutuhan di mana FR cenderung lebih dominan.

\subsection{Protokol Anotasi}
Setiap kalimat dianotasi oleh dua validator dengan pengalaman minimal 3 tahun di bidang Rekayasa Perangkat Lunak (RPL). Protokol anotasi meliputi:
\begin{enumerate}
    \item Pelabelan manual berbasis pedoman standar ISO/IEC 29148.
    \item Validasi silang pada 20\% sampel dataset.
    \item Reliabilitas \textit{inter-annotator} diuji menggunakan Cohen's Kappa dengan nilai 0.72 (kategori \textit{substantial agreement}).
\end{enumerate}

\section{Pra-pemrosesan Data} \label{III.Preprocessing}
Penelitian menerapkan pendekatan \textbf{Split-Preprocessing} untuk menyesuaikan kebutuhan input arsitektur model yang berbeda.

\subsection{Pra-pemrosesan Umum}
Langkah umum yang diterapkan pada seluruh data meliputi:
\begin{enumerate}
    \item Pembersihan teks: penghapusan URL, tag HTML, karakter non-ASCII, dan simbol yang tidak relevan.
    \item \textit{Case folding} menjadi huruf kecil.
    \item Normalisasi spasi dan tanda baca.
\end{enumerate}

\subsection{Pipeline A: SVM}
Untuk SVM, digunakan pendekatan reduksi fitur agresif (Gambar \ref{fig:3.flow_svm}):
\begin{enumerate}
    \item \textit{Stopword removal} menggunakan pustaka Sastrawi.
    \item \textit{Stemming} menggunakan algoritma Nazief–Adriani.
    \item Tokenisasi berbasis \textit{whitespace}.
\end{enumerate}

\begin{figure}[!ht]
    \centering
    \includegraphics[width=0.95\textwidth, height=0.60\textheight, keepaspectratio]{figure/flow_svm.png}
    \caption{Pipeline Pra-pemrosesan dan Pelatihan SVM}
    \label{fig:3.flow_svm}
\end{figure}

\subsection{Pipeline B: Bi-LSTM dan DistilBERT}
Untuk menjaga konteks semantik yang dibutuhkan model \textit{Deep Learning}:
\begin{enumerate}
    \item \textit{Stopword removal} dan \textit{stemming} \textbf{tidak diterapkan}.
    \item Tokenisasi:
        \begin{enumerate}
            \item \textbf{Bi-LSTM}: \textit{Word-level tokenization}.
            \item \textbf{DistilBERT}: \textit{WordPiece tokenizer} (sub-word).
        \end{enumerate}
    \item \textit{Padding} dan \textit{truncation} pada panjang tetap 128 token.
\end{enumerate}

\section{Representasi Fitur} \label{III.Fitur}

\subsection{TF-IDF untuk Support Vector Machine (SVM)}
TF-IDF (\textit{Term Frequency-Inverse Document Frequency}) dipilih karena efisiensi komputasi dan interpretabilitasnya [Salton \& Buckley, 1988]. Metode ini menghasilkan vektor jarang (\textit{sparse vector}) yang stabil untuk model linear.

TF-IDF untuk term $t$ dan dokumen $d$ didefinisikan secara matematis sebagai:

\begin{equation}
    \text{TF-IDF}(t, d) = \text{TF}(t, d) \times \text{IDF}(t)
    \label{eq:tfidf}
\end{equation}

Dimana $\text{TF}(t, d)$ adalah frekuensi term ternormalisasi:

\begin{equation}
    \text{TF}(t, d) = \frac{n_{t,d}}{\sum_{k} n_{k,d}}
\end{equation}

Dan IDF mengukur informasi yang dibawa oleh kata tersebut:

\begin{equation}
    \text{IDF}(t) = \log\left(\frac{N}{df_t}\right) + 1
\end{equation}

Dimana $N$ adalah total dokumen (1.000) dan $df_t$ adalah jumlah dokumen yang mengandung term $t$.

\textbf{Konfigurasi Parameter:}
\begin{enumerate}
    \item \texttt{max\_features}: 5.000 (mengambil 5.000 kata terpenting).
    \item \texttt{ngram\_range}: (1, 2) (unigram dan bigram).
\end{enumerate}

\subsection{Trainable Embedding untuk Bi-LSTM}
Penelitian ini menggunakan \textit{Trainable Embedding} yang dilatih dari awal (\textit{from scratch}). Keputusan ini didasarkan pada kebutuhan **adaptasi domain spesifik**. Kosakata dalam dokumen *requirements* memiliki distribusi makna yang berbeda dengan korpus bahasa umum (seperti Wikipedia).

Dengan 1.000 sampel data latih, model embedding dimensi 100 ($d=100$) masih dapat konvergen dengan baik tanpa \textit{overfitting} berlebihan, dibandingkan menggunakan \textit{pre-trained} embedding dimensi 300 yang terlalu kompleks untuk tugas ini.

Representasi setiap token input $w_t$ diperoleh melalui operasi *lookup*:

\begin{equation}
    \mathbf{e}_t = E[w_t] \in \mathbb{R}^{100}
\end{equation}

Matriks $E$ diperbarui selama proses *backpropagation*.

\subsection{Contextual Embedding untuk DistilBERT}
DistilBERT menggunakan mekanisme \textit{Self-Attention} untuk menghasilkan representasi kontekstual. Vektor representasi diambil dari \textit{hidden state} terakhir pada token spesial \texttt{[CLS]}, yang merepresentasikan semantik kalimat secara global.

\section{Arsitektur dan Implementasi Model} \label{III.Model}

\subsection{Support Vector Machine (SVM)}
SVM bekerja dengan mencari \textit{hyperplane} optimal yang memisahkan kelas FR dan NFR. Formulasi optimasi primal SVM [Vapnik, 1995] adalah:

\begin{equation}
    \min_{\mathbf{w}, b, \xi} \quad \frac{1}{2}\|\mathbf{w}\|^2 + C\sum_{i=1}^n \xi_i
\end{equation}

Dengan batasan:
\begin{equation}
    y_i(\mathbf{w}^\top \phi(\mathbf{x}_i) + b) \geq 1 - \xi_i
\end{equation}

Penelitian ini menggunakan Kernel RBF (\textit{Radial Basis Function}):
\begin{equation}
    K(\mathbf{x}_i, \mathbf{x}_j) = \exp\left(-\gamma \|\mathbf{x}_i - \mathbf{x}_j\|^2\right)
\end{equation}

\textbf{Konfigurasi Hyperparameter:}
\begin{table}[!ht]
\centering
\caption{Konfigurasi Hyperparameter SVM}
\begin{tabular}{|l|l|}
\hline
\textbf{Parameter} & \textbf{Nilai} \\ \hline
Kernel & RBF \\ \hline
C & 1.0 \\ \hline
Gamma & Scale \\ \hline
Class Weight & Balanced \\ \hline
\end{tabular}
\end{table}

Penggunaan \texttt{class\_weight='balanced'} secara otomatis menyesuaikan bobot penalti berbanding terbalik dengan frekuensi kelas untuk mengatasi ketidakseimbangan data (FR 58\% vs NFR 42\%).

\subsection{Bidirectional LSTM (Bi-LSTM)}
Bi-LSTM memproses sekuens input dari dua arah menggunakan mekanisme \textit{gating} [Hochreiter \& Schmidhuber, 1997]. Untuk setiap langkah waktu $t$, persamaan pembaruan sel adalah sebagai berikut:

% Rumus 1: Forget Gate
\begin{equation}
    f_t = \sigma(W_{f} \cdot [h_{t-1}, x_t] + b_f)
\end{equation}

% Rumus 2: Input Gate
\begin{equation}
    i_t = \sigma(W_{i} \cdot [h_{t-1}, x_t] + b_i)
\end{equation}

% Rumus 3: Cell Candidate
\begin{equation}
    \tilde{C}_t = \tanh(W_{C} \cdot [h_{t-1}, x_t] + b_C)
\end{equation}

% Rumus 4: Cell State Update
\begin{equation}
    C_t = f_t * C_{t-1} + i_t * \tilde{C}_t
\end{equation}

% Rumus 5: Output Gate
\begin{equation}
    o_t = \sigma(W_{o} \cdot [h_{t-1}, x_t] + b_o)
\end{equation}

% Rumus 6: Hidden State
\begin{equation}
    h_t = o_t * \tanh(C_t)
\end{equation}

Keluaran akhir diperoleh dengan menggabungkan (\textit{concatenate}) state terakhir dari arah maju ($\rightarrow$) dan mundur ($\leftarrow$):

\begin{equation}
    \mathbf{h}_{\text{final}} = \text{Concat}(\mathbf{h}^{\rightarrow}_T, \mathbf{h}^{\leftarrow}_T)
\end{equation}

\textbf{Penanganan Ketidakseimbangan Data:}
Berbeda dengan SVM yang menggunakan parameter bobot, pada Bi-LSTM diterapkan \textbf{Weighted Binary Cross-Entropy Loss}. Fungsi *loss* dimodifikasi dengan bobot $w_j$ yang diberikan lebih besar untuk kelas minoritas (NFR) agar model tidak bias ke kelas mayoritas.

\begin{equation}
    \mathcal{L} = -\frac{1}{N} \sum_{i=1}^{N} w_{y_i} \left[ y_i \log(\hat{y}_i) + (1 - y_i) \log(1 - \hat{y}_i) \right]
\end{equation}

\textbf{Konfigurasi Hyperparameter:}
Nilai hyperparameter ditentukan melalui \textbf{studi empiris awal} (\textit{preliminary experiments}) pada data validasi untuk mencapai konvergensi yang stabil.

\begin{table}[!ht]
\centering
\caption{Hyperparameter Bi-LSTM}
\label{tab:lstm_params}
\begin{tabular}{|l|l|}
\hline
\textbf{Parameter} & \textbf{Nilai} \\ \hline
Embedding Dim & 100 \\ \hline
LSTM Units & 64 (Bidirectional) \\ \hline
Dropout Rate & 0.5 \\ \hline
Optimizer & Adam ($lr=1e-3$) \\ \hline
Batch Size & 32 \\ \hline
Epochs & 10 (Early Stopping) \\ \hline
L2 Regularization & 0.001 \\ \hline
\end{tabular}
\end{table}

\begin{figure}[!ht]
    \centering
    \includegraphics[width=0.95\textwidth, height=0.60\textheight, keepaspectratio]{figure/flow_lstm.png}
    \caption{Arsitektur Bi-LSTM}
    \label{fig:3.flow_lstm}
\end{figure}

\subsection{DistilBERT (Fine-Tuning)}
DistilBERT adalah varian ringan dari BERT yang menggunakan mekanisme \textit{Knowledge Distillation} [Sanh et al., 2019]. Inti dari model ini adalah \textit{Multi-Head Self-Attention}:

\begin{equation}
    \text{Attention}(Q, K, V) = \text{softmax}\left(\frac{QK^T}{\sqrt{d_k}}\right)V
\end{equation}

Proses \textit{fine-tuning} dilakukan dengan menambahkan satu lapisan klasifikasi (\textit{classification head}) di atas output token \texttt{[CLS]}.

\textbf{Penanganan Ketidakseimbangan Data:}
Sama halnya dengan Bi-LSTM, pelatihan DistilBERT menggunakan \textbf{WeightedLoss} pada objek \textit{Trainer} dari pustaka HuggingFace, untuk menyeimbangkan kontribusi gradien dari kelas NFR.

\textbf{Konfigurasi Hyperparameter:}
\begin{table}[!ht]
\centering
\caption{Konfigurasi Fine-tuning DistilBERT}
\label{tab:distilbert_params}
\begin{tabular}{|l|l|}
\hline
\textbf{Parameter} & \textbf{Nilai} \\ \hline
Base Model & distilbert-base-multilingual-cased \\ \hline
Learning Rate & $2 \times 10^{-5}$ \\ \hline
Batch Size & 16 \\ \hline
Epochs & 3 -- 5 \\ \hline
Optimizer & AdamW \\ \hline
Weight Decay & 0.01 \\ \hline
Warmup Steps & 100 \\ \hline
\end{tabular}
\end{table}

\begin{figure}[!ht]
    \centering
    \includegraphics[width=0.95\textwidth, height=0.60\textheight, keepaspectratio]{figure/flow_distilbert.png}
    \caption{Alur Fine-tuning DistilBERT}
    \label{fig:3.flow_distilbert}
\end{figure}

\section{Skenario Pengujian} \label{III.Skenario}

\subsection{Pembagian Data}
\begin{enumerate}
    \item Rasio Data Latih/Uji: 80:20.
    \item Metode: \textit{Stratified Sampling} (mempertahankan rasio kelas).
    \item \textit{Random Seed}: 42 (untuk reproduktifitas).
\end{enumerate}

\subsection{Stratified 5-Fold Cross Validation}
Pengujian dilakukan menggunakan validasi silang 5-lipatan yang terstratifikasi. Hasil akhir merupakan rata-rata dari 5 iterasi pengujian untuk mengurangi bias varians pada estimasi performa.

\section{Evaluasi dan Validasi} \label{III.Evaluasi}

\subsection{Metrik Performa}
Evaluasi didasarkan pada elemen \textit{Confusion Matrix} (TP, TN, FP, FN).

\textbf{Precision:}
\begin{equation}
    \text{Precision} = \frac{TP}{TP + FP}
\end{equation}

\textbf{Recall:}
\begin{equation}
    \text{Recall} = \frac{TP}{TP + FN}
\end{equation}

\textbf{F1-Score:}
Metrik utama yang digunakan adalah F1-Score karena memberikan keseimbangan antara presisi dan recall, terutama pada dataset yang tidak seimbang sempurna.

\begin{equation}
    \text{F1-Score} = 2 \times \frac{\text{Precision} \times \text{Recall}}{\text{Precision} + \text{Recall}}
\end{equation}

\section{Analisis Kesalahan} \label{III.ErrorAnalysis}
Analisis kesalahan (\textit{Error Analysis}) dilakukan secara kualitatif dengan memeriksa sampel \textit{False Positive} dan \textit{False Negative}. Tujuannya adalah mengidentifikasi pola linguistik spesifik (seperti ambiguitas, negasi, atau struktur kalimat kompleks) yang gagal diprediksi dengan benar oleh model.

\section{Lingkungan Pengembangan} \label{III.Lingkungan}
Seluruh eksperimen dijalankan pada lingkungan komputasi awan Google Colab dengan spesifikasi:
\begin{enumerate}
    \item \textbf{GPU}: NVIDIA Tesla T4 (16GB VRAM) untuk akselerasi pelatihan Deep Learning.
    \item \textbf{RAM}: 12 GB.
    \item \textbf{Framework}: TensorFlow 2.15, PyTorch 2.1, HuggingFace Transformers.
    \item \textbf{Bahasa}: Python 3.10.
\end{enumerate}
